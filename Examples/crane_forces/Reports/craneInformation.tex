\title{Crane Model}
\author{
	Kenneth Kuchera
}
\date{\today}
\documentclass[12pt]{article}
\usepackage{amsmath}
\usepackage[margin=1in]{geometry}

\begin{document}
\maketitle

\section{Model}
This example is located in \emph{svn}/\emph{software}/\emph{examples}/\emph{simulink}/\emph{crane}/\emph{Branch}/\emph{v0.5}. The model used is described in deliverable 8.2. We will start by restating the model equations here. We will use these equations which describe a non-linear continuous model to derive a linear discrete model. The matrices that are used in the simulation as well as the process of linearizing, discretizing, can be found in the file \emph{craneInformation.m}.

\subsection{Non-Linear Continuous Model}
The non linear model is given by the following equations.
\begin{equation}
\begin{aligned}
	\dot{x}_C &= v_C \\
	\dot{x}_L &= v_L \\
	\dot{v}_C &= \frac{-1}{\tau_C}\*v_C + \frac{A_C}{\tau_C}\*u_C \\
	\dot{v}_L &= \frac{-1}{\tau_L}\*v_L + \frac{A_L}{\tau_L}\*u_L \\
	\dot{\theta} &= \omega \\
	\dot{\omega}_T &= - \frac{1}{x_L} \left( g\*sin(\theta) + \left( \frac{-1}{\tau_C} \* v_C + \frac{A_C}{\tau_C} \*u_C \right) \* cos(\theta) + 2\*v_L\*\omega + \frac{c\*\omega}{m\*x_L} \right) \\
	\dot{u}_{C} &= u_{CR} \\
	\dot{u}_{L} &= u_{LR} \\
\end{aligned}
\label{eq:NLCM}
\end{equation}

In a first step we will linearize this. In a second step we will discretize it. This is because we want to use a linear discrete model for the MPC instead of a non-linear continuous model.

\subsubsection{Linear Continuous Model}
Linearizing the model from equation~\ref{eq:NLCM} around a steady state yields an equation of the form $\dot{x} = \mathbf{A_c} \* \left( x - x^* \right) + \mathbf{B_c} \* \left( u - u^* \right)$. Expanding this equation gives

\begin{equation}
\underbrace{
\begin{bmatrix}
\dot{x}_C \\
\dot{x}_L \\
\dot{v}_C \\
\dot{v}_L \\
\dot{\theta} \\
\dot{\omega} \\
\dot{u}_C \\
\dot{u}_L \\
\end{bmatrix}
}_{\dot{x}}
= 
\underbrace{
\begin{bmatrix}
0   &    0   &    1   &    0   &    0   &    0   &    0   &    0 \\
0   &    0   &    0   &    1   &    0   &    0   &    0   &    0 \\
0   &    0   &\frac{-1}{\tau_C}&    0    &   0    &   0 &  \frac{A_C}{\tau_C} &  0 \\
0   &    0   &    0  &  \frac{-1}{\tau_L} &  0   &    0   &    0  &  \frac{A_L}{\tau_L} \\
0   &    0   &    0  &     0   &    0   &    1   &    0   &    0 \\
0   &  \frac{\partial{\dot{\omega}}}{\partial{x_L}}& \frac{\partial{\dot{\omega}}}{\partial{v_C}}&   \frac{\partial{\dot{\omega}}}{{v_L}} & \frac{\partial{\dot{\omega}}}{\partial{\theta}} & \frac{\partial{\dot{\omega}}}{\partial{\omega}} & \frac{\partial{\dot{\omega}}}{\partial{u_C}} &   0 \\
0   &    0   &    0  &     0   &    0   &    0   &    0   &    0 \\
0   &    0   &    0   &    0   &    0   &    0    &   0   &    0 \\
\end{bmatrix}
}_\mathbf{A_c}
\underbrace{
\begin{bmatrix}
x_C - x_C^* \\
x_L - x_L^* \\
v_C - v_C^* \\
v_L - v_L^* \\
\theta - \theta^*\\
\omega - \omega^*\\
u_C - u_C^*\\
u_L - u_L^*\\
\end{bmatrix}
}_{(x-x^*)}
+
\underbrace{
\begin{bmatrix}
0 & 0 \\
0 & 0 \\
0 & 0 \\
0 & 0 \\
0 & 0 \\
0 & 0 \\
1 & 0 \\
0 & 1 \\
\end{bmatrix}
}_\mathbf{B_c}
\underbrace{
\begin{bmatrix}
u_{CR} - u_{CR}^* \\
u_{LR} - u_{LR}^* \\
\end{bmatrix}
}_{(u-u^*)}
\label{eq:LCM}
\end{equation}
with 
\begin{equation*}
\begin{aligned}
\frac{\partial{\dot{\omega}}}{\partial{x_L}} &= \frac{1}{x_L^2} \* \left( g\*sin(\theta) + \left( \frac{-1}{\tau_C} \* v_C + \frac{A_C}{\tau_C} \* u_C \right) \* cos(\theta)) + 2\*v_L\*\omega + 2\*\frac{c\*\omega}{m\*x_L} \right)\\
\frac{\partial{\dot{\omega}}}{\partial{v_C}} &= \frac{cos(\theta)}{\tau_C\*x_L} \\
\frac{\partial{\dot{\omega}}}{\partial{v_L}} &= \frac{-2\*\omega}{x_L} \\
\frac{\partial{\dot{\omega}}}{\partial{\theta}} &= \frac{1}{x_L} \* \left( - g\*cos(\theta) + \left( -\frac{v_C}{\tau_C} + \frac{A_C}{\tau_C}\*u_C \right)\*sin(\theta) \right) \\
\frac{\partial{\dot{\omega}}}{\partial{\omega}} &= \frac{-2\*v_L}{x_L} \\
\frac{\partial{\dot{\omega}}}{\partial{u_C}} &= \frac{-A_C\*cos(\theta)}{\tau_C\*x_L} \\
\end{aligned}
\end{equation*}

\subsection{Linear Discrete Model}
The discretization takes place by constructing a matrix M, taking the exponent and finally extracting the matrices for the discrete model. This is done as follows.
\begin{equation*}
\begin{bmatrix}
\mathbf{A_d} & \mathbf{B_d} \\
\mathbf{0_{n_u \times n_x}} & \mathbf{I_{n_u \times n_u}}
\end{bmatrix}
=
exp\left(
\begin{bmatrix}
\mathbf{A_c} \* t_s & \mathbf{B_c}\*t_s \\
\mathbf{0_{n_u \times n_x}} & \mathbf{0_{n_u \times n_u}} \\
\end{bmatrix}.
\right)
\end{equation*}

Where $t_s$ is the sampling rate, $\mathbf{A_d}$ and $\mathbf{B_d}$ the matrices $\mathbf{A}$ and $\mathbf{B}$ respectively for the discrete system.

\subsection{Implementation}
The values used in the implementation are listed in the tables below. They can also be found in the file \emph{craneInformation.m} which is located in the repository.
\begin{table}[!ht]
\centering
\begin{tabular}{| l | l | l |}
\hline
\multicolumn{3}{|c|}{\textbf{Constants}} \\
\hline
Constant & Value & Description \\
\hline 
$m$   &    $1318$ & mass $(g)$ \\
$g$   &    9.81 & gravity $(\frac{m}{s^2})$ \\
$c$   &    0    &             damping $(\frac{kgm^2}{s})$ \\
$A_C$ &    0.047418203070092 & Gain of cart vel. ctrl. $(\frac{m}{s}V)$ \\
$\tau_C$ & 0.012790605943772 & time constant cart ctrl. $(s)$ \\
$A_L$ &    0.034087337273386 & Gain of winch vel. ctrl. $(\frac{m}{s}V)$ \\
$\tau_L$ & 0.024695192379264 & time constant winch ctrl. $(s)$ \\
\hline
\end{tabular}
\end{table}

\begin{table}[!ht]
\centering
\begin{tabular}{| l | l | l |}
\hline
\multicolumn{3}{|c|}{\textbf{Working point}} \\
\hline
Working point & Value & Description \\
\hline 
$x_C^*$ & 1    & cart position $(m)$ \\
$x_L^*$ & 1.5  & cable length $(m)$ \\
$v_C^*$ & 0    & cart velocity $(\frac{m}{s})$ \\
$v_L^*$ & 0    & cable velocity $(\frac{m}{s})$ \\
$\theta^*$ & 0 & cable deflection $(rad)$ \\
$\omega^*$ & 0 & cable deflection rate $(\frac{rad}{s})$ \\
$u_C^*$ & 0 & cart controller voltage $(V)$ \\
$u_L^*$ & 0 & cable controller voltage $(V)$ \\
\hline
\end{tabular}
\end{table}

\begin{table}[!ht]
\centering
\begin{tabular}{| l | l | l |}
\hline
\multicolumn{3}{|c|}{\textbf{Initial point}} \\
\hline
Initial point & Value & Description \\
\hline 
$x_C$ & 0    & cart position $(m)$ \\
$x_L$ & 0.75 & cable length $(m)$ \\
$v_C$ & 0    & cart velocity $(\frac{m}{s})$ \\
$v_L$ & 0    & cable velocity $(\frac{m}{s})$ \\
$\theta$ & 0 & cable deflection $(rad)$ \\
$\omega$ & 0 & cable deflection rate $(\frac{rad}{s})$ \\
$u_C$ & 0 & cart controller voltage $(V)$ \\
$u_L*$ & 0 & cable controller voltage $(V)$ \\
\hline
\end{tabular}
\end{table}

Note: The point around which we have linearized (working point) is also the setpoint.

\newpage
\section{Model Predictive Control}
A standard MPC algorithm is used with the following costs.
\begin{equation}
\begin{aligned}
Q &= diag(100, 100, 100, 100, 100, 100, 1, 1) \\
R &= diag(1,1) \\
P &= \text{solution to the Ricatti equation for the discrete system, see craneInformation.m}
\end{aligned}
\end{equation}

Here $Q$ is the cost on the states, $R$ is the cost on the inputs and $P$ is the terminal cost on the states.

\end{document}
